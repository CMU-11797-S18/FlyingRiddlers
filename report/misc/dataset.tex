
\section{Dataset}

We build our QA systems on two datasets: BioASQ dataset (from PubMed documents) and MAchine Reading COmprehension (MS MARCO) dataset (from web documents).

\subsection{BioASQ}
\label{Dataset}

BioASQ dataset is 
BioASQ challenge \cite{bioasq} is a large scale biomedical question answering and semantic indexing challenge, which has been running as an annual competition since 2013. %This challenge assesses the ability of systems to semantically index very large numbers of biomedical scientific articles, and to return concise and user-understandable answers to given natural language questions by combining information from biomedical articles and ontologies.
We deal with the Phase B of the challenge which deals with large scale biomedical question answering. The dataset provides a set of questions and snippets from PubMed, which are relevant to the specific question. It also provides users with a question type and urls of the relevant PubMed articles itself. The 5b version of this dataset consists of 1,799 questions in 3 distinct categories:
\begin{enumerate}
    \item \textbf{Factoid type}: This question type has a single entity as the ground truth answer and expects the systems to output a set of  entities ordered by relevance; systems are evaluated using the mean reciprocal rank \cite{MRR} of the answer entities with reference to the ground truth answer entity.
    \item \textbf{List type}: This answer type expects the system to return an unordered list of entities as answer and evaluates them using a F-score based metric against a list of reference answer entities which can vary in number.
    \item \textbf{Yes/No type}: This question type asks the systems to answer a given question with a binary output namely yes or no. The questions typically require reasoning and inference over the evidence snippets to be able to answer the questions correctly.
\end{enumerate}

The dataset expected the participants to generate two types of answers, namely, exact and ideal answers. 
In ideal answers, the systems are expected to generate a well formed paragraph for each of the question types which explains the answer to the question. They call these answers `ideal' because it is what a human would expect as an answer by a peer biomedical scientist. In the exact answers the systems are expected to generate ``yes" or ``no" in the case of yes/no questions, named entities in the case of factoid questions and list of named entities in the case of list questions.

\subsection{MS MARCO}